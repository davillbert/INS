\chapter*{\centering\large{Заключение}}
\addcontentsline{toc}{chapter}{Заключение}		
    \label{sec:Chapter4} \index{Chapter4}

 \begin{onehalfspace}
В результате исследования для процесса передачи данных были определены факторы, влияющие на точность программно-имитационного моделирования и определены подходы к расчёту временной погрешности моделирования процесса, а также получены зависимости временных задержек передачи данных, позволяющие увеличить точность имитационной модели канала связи. Это позволяет более эффективно анализировать и оптимизировать системы связи, которые могут быть представлены с помощью блоков модели.

Блоки, используемые при разработке данной модели канала связи, могут быть настроены под исследуемые задачи, такие как внешнее воздействие среды на передачу сигнала, то есть на время передачи и целостность полученной информации относительно переданной. Симуляция радиоканалов системы (к примеру, <<ретранслятор - абонент>>) может быть использована для отработки моделирования задержки передачи данных, возникающих из-за таких факторов, как многолучевое распространение сигнала, потери сигнала, интерференцию, эффекты перемещения объектов, шумы и искажения и другие.
В результате исследования на примере рассматриваемого процесса ретрансляции была разработана имитационная модель «базовая станция — ретранслятор — абонент», были получены зависимости временных задержек передачи данных, позволяющие увеличить точность имитационной модели канала связи. Задержки могут возникать в различных частях канала радиосвязи, включая передачу сигнала по каналу от передатчика к приемнику, что является внешним фактором, проявляющимся из-за влияния физической среды, обработку сигнала в приемнике, также влияние оказывает время затрачиваемое на обработку данных в соответствие с протоколом. В результате были отмечены и рассмотрены следующие факторы:
 	
	\begin{enumerate} 	
	\item	Задержки передачи сигнала, связанные с модуляцией, могут возникать из-за различных факторов, включая характеристики самого модулятора и особенности передающей среды.
	
	\item	Задержки передачи радиосигнала могут возникать из-за нескольких факторов, связанных с фильтром, в особенности из-за того, что фильтр работает на основе обработки сигнала во временной области и изменяет его форму. В процессе обработки фильтр создает временные запаздывания и изменения фазы, что приводит к задержке в передаче. 


    \item	Задержки в среде распространения, сопутствующие затухания, помехи и переотражения. 
 
	\end{enumerate}
 



 
Направлениями дальнейших исследований являются проверка модели на прототипе системы передачи данных, а также оптимизация алгоритмов передачи данных в части временных задержек. Планируется расширить модель на многоабонентные системы связи. Расширение модели на многоабонентные системы связи позволит более точно моделировать и анализировать процессы передачи данных в таких системах, что открывает новые возможности для улучшения и оптимизации сетевых решений. Сравнение результатов имитационной модели с реальными данными из радиосвязи может помочь выявить расхождения и определить области для улучшения модели или наоборот, отработки оборудования до теоретических пределов.
 

	



 
\end{onehalfspace} 
	\newpage



