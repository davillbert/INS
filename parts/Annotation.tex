\large{\begin{abstract}


\begin{onehalfspace}



    Бесплатформенные инерциальные навигационные системы традиционно применяются в составе бортового оборудования широкого класса подвижных объектов — самолетов и вертолетов, судов и космических аппаратов.
    
    	В настоящей работе рассматриваются способы работы с данными, получаемыми с помощью навигационных систем, которые могут быть использоваться в системах управления подвижными объектами. Основная функция навигационной системы — вычисление положения, скорости и ориентации объекта относительно координатных систем независимо от движения объекта, с помощью численного моделирования движения можно исследовать возможности сбора и обрабокт информации БИНС и внешних факторов. 
     
     ИНС дает информацию о второстепенных параметрах движения, вычисляемых по основной навигационной информации или же с привлечением внешних данных, которые формируют оценку скорости и положения в пространстве. Также, зная основные параметры и их ошибки, всегда можно найти второстепенные параметры и их ошибки. Например, по заданному вектору скорости в географических осях взятому за основной параметр можно вычислить путевой угол (второстепенный параметр), а по ошибкам скорости — ошибка путевого угла. 
     
     Для оценки как основных, так и следующих из них второстепенных параметров, используются специальные приборы и датчики.  В зависимости от решаемых задач, ИНС имеют различные классы точности, низкий класс точности — МЭМС, высокий класс точности — ВТГ и кварцевых акселерометров.

     азработкой и производством навигационных систем заняты подразделения крупных промышленных  корпораций, таких как Honeywell, Northrop Grumman, Safran, Thales. В России разработкой и выпуском ИНС занимаются ПАО «МИЭА», АО ИТТ, «АО «Концерн «ЦНИИ «Элетроприбор», НПК «Оптолик», НПК «Электрооптика», ПНППК и другие производители. 
     
 МЭМС — устройства типа микроэлектромеханических систем, в них используются подвижные массы в качестве чувствительных элементов. Производством подобного оборудования занимаются компании Sagem, Litef, чьи устройства позволяют достигать класса точности $150^{\circ}/$час$ - 1^{\circ}/$час и $10 - 1 mg$. К устройствам более выского класса точности относятся кольцевой лазер (RLG) и волоконно-оптические гироскопы (FOG). Оба прибора работают на основе общего принципа - эффекта Саньяка. Они могут достигать значений до $0.01^{\circ}/$час и до $100 \mu g$.  Французская компания Thales разрабатывает и производит оборужование даннаго типа на собственных предприятиях. 

    Наиболее точными являются приборы FIBER-OPTIC GYROSCOPE HP, способные достигать инструментальных точностей до $0.001^{\circ}/$час и до $10 \mu g$. Однако без применения алгоритмов обработки данных, получаемых даже с самых точных приборов, данные о движение будут искажены относительно реальности. В таком случае необходима фильтрация и раздельные независимые измерения акселерометров и гироскопов, чтобы сгладить различие между результатом навигационных вычислений и истинной траекторией. Так, коррекция на базе фильтра Калмана обеспечивает калибровку гироскопов в режиме реального времени.


    Одним из способов повышения точности инерциального режима навигации автокалибровка на основе модельных данных – повышение 
за счет использования оценок погрешностей инерциальных датчиков, полученных в предыдущих результате моделирования (имитации, симуляции) работы аппарата. Для решения задачи коррекции используются алгоритмы оценивания
калмановского типа с устойчивой численной реализацией дискретного фильтра.

   

\end{onehalfspace}



\end{abstract}}
